\chapter*{Kết luận}
\label{chap:conclusion}

Trong quá trình nghiên cứu, chúng ta đã lần lượt tiếp cận và phân tích 
nhiều phương pháp lặp khác nhau được sử dụng trong giải tích số — từ các kỹ thuật tìm nghiệm của phương trình phi tuyến một ẩn đến các phương pháp lặp giải hệ phương trình tuyến tính nhiều ẩn.  
Những phương pháp này không chỉ đóng vai trò là công cụ tính toán hiệu quả, mà còn thể hiện tư tưởng nền tảng của toán học tính toán hiện đại: 
\textit{dùng quá trình xấp xỉ có kiểm soát để tiệm cận dần đến nghiệm chính xác}.

Ở phần đầu, các phương pháp như Bisection, Fixed-Point Iteration, Newton và các biến thể của nó cho thấy sự tiến hóa từ những kỹ thuật đơn giản, ổn định đến những phương pháp có tốc độ hội tụ nhanh và độ chính xác cao hơn.
Phân tích sai số và bậc hội tụ giúp ta hiểu rõ hơn vì sao cùng một bài toán, các phương pháp khác nhau lại cho kết quả và tốc độ tiệm cận khác nhau.  
Những kỹ thuật tăng tốc hội tụ như Aitken’s $\Delta^2$ Process và Steffensen’s Method minh chứng cho khả năng kết hợp giữa tính hiệu quả và sự tinh gọn về mặt thuật toán, cho phép đạt được tốc độ hội tụ bậc hai mà không cần đạo hàm — một bước tiến quan trọng trong thiết kế các phương pháp lặp hiện đại.

Phần sau mở rộng các tư tưởng đó sang lĩnh vực đại số tuyến tính,
nơi các hệ phương trình lớn xuất hiện phổ biến trong kỹ thuật, vật lý và mô phỏng số.
Các phương pháp Jacobi, Gauss–Seidel và SOR (Successive Over-Relaxation) 
thể hiện cách tiếp cận linh hoạt trong việc giải hệ phương trình tuyến tính thông qua quá trình lặp dần, đồng thời cung cấp cơ chế kiểm soát sai số và tăng tốc hội tụ dựa trên cấu trúc ma trận và tham số thư giãn thích hợp.
Các kỹ thuật đánh giá sai số và hiệu chỉnh nghiệm lặp (Iterative Refinement)cũng được giới thiệu như một phần không thể thiếu để đảm bảo tính ổn định và độ chính xác của kết quả tính toán.

Tổng thể, các nội dung được trình bày trong tiểu luận 
đã cho thấy mối liên hệ chặt chẽ giữa lý thuyết và thực hành trong giải tích số.  
Các phương pháp lặp không chỉ là công cụ giải toán thuần túy, 
mà còn là nền tảng cho hàng loạt thuật toán hiện đại 
trong tối ưu hóa, học máy, mô phỏng động lực học và tính toán khoa học.

